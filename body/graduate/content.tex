\chapter{绪论}

本模板根据浙江大学研究生院编写的《浙江大学研究生学位论文编写规则》~\cite{zjugradthesisrules},
在原有的 zjuthesis 模板~\cite{zjuthesis}基础上开发而来。

本模板的本科生版本\cite{zjuthesisrules}得到了浙江大学本科生院老师的支持与审核,
已经在本科生院网上公示。
但当前的研究生版本并未经过研究生院老师的审核,
同学们使用时要注意对照模板与要求,
切不可盲目使用。

作者本人并未编写过浙江大学研究生毕业论文,
所以不清楚具体要求。
如果有热心同学愿意帮忙,
可以替我联系相关老师,我会配合审核并修改代码。

\section{背景}

% 如果你在Overleaf上编译本模板,请注意如下事项:

% \begin{itemize}
%     \item 删除根目录的 ``.latexmkrc'' 文件,否则编译失败且不报任何错误
%     \item 字体有版权所以本模板不能附带字体,请务必手动上传字体文件,并在各个专业模板下手动指定字体。
%         具体方法参照 GitHub 主页的说明。
%     \item 当前的Overleaf默认使用TexLive 2017进行编译,但一些伪粗体复制乱码的问题需要TexLive 2019版本来解决。
%         所以各位同学可以在Overleaf上编写论文时务必切换到TexLive 2019或更新版本来编译,以免产生查重相关问题。
%         具体说明参照 GitHub 主页。
% \end{itemize}


\section{研究现状}

% 我们可以用includegraphics来插入现有的jpg等格式的图片,
% 如\autoref{fig:zju-logo}所示。

\begin{figure}[htbp]
    \centering
    \includegraphics[width=.3\linewidth]{logo/zju}
    \caption{\label{fig:zju-logo}浙江大学LOGO}
\end{figure}

重点突出目前的研究处于什么状态,尚存在的问题

\subsection{体数据相关性分析}
\subsection{体数据特征提取}
\subsection{体数据表征学习}
\section{本文工作及组织结构}


\par 如\autoref{tab:sample}所示,这是一张自动调节列宽的表格。

\begin{table}[htbp]
    \caption{\label{tab:sample}自动调节列宽的表格}
    \begin{tabularx}{\linewidth}{c|X<{\centering}}
        \hline
        第一列 & 第二列 \\ \hline
        xxx & xxx \\ \hline
        xxx & xxx \\ \hline
        xxx & xxx \\ \hline
    \end{tabularx}
\end{table}


\par 如\autoref{equ:sample},这是一个公式

\begin{equation}
    \label{equ:sample}
    A=\overbrace{(a+b+c)+\underbrace{i(d+e+f)}_{\text{虚数}}}^{\text{复数}}
\end{equation}

\chapter{体数据嵌入表征的学习与应用}

\section{传统的体数据信息表达(TODO:在后面提及,本节不写)}
\subsection{基于数值统计的信息表达方法(TODO:在第三章、第四章提及)}
通过挖掘体数据内部的属性,如标量值、梯度、统计学等信息对数据进行深层次表达

【数据压缩】

【统计分布】

\subsection{基于拓扑结构的信息表达方法(在第五章提及)}

通过树、图等数据组织方式对数据进行抽象表征(TODO:暂不提及)

【contour tree】

【Reeb Graph】

【nested tracking graphs (2017)】

【Dynamic nested tracking graphs (2019)】

【fiber surfaces: Generalizing isosurfaces to bivariate data (2015)】

【Transgraph: Hierarchical exploration of transition relationships in time-varying volumetric data (2011)】

【persistence diagram】

【Graphs in scientific visualization: A survey (2017)】

【Joint contour nets (2014)】


\section{体数据相关性分析}
\subsection{体素相关性}
\subsection{数值相关性}
\subsection{体数据整体相关性}
\subsection{特征相关性}

【Transgraph: Hierarchical exploration of transition relationships in time-varying volumetric data (2011)】

\section{体数据特征提取}
\subsection{单变量体数据特征提取}

基于传统方式的特征提取方法

【Texture-based transfer functions for direct volume rendering (2008)】

【State of the art in transfer functions for direct volume rendering (2016)】

基于机器学习的特征提取方法

【Semi-automatic generation of transfer functions for direct volume rendering (1998)】

【An intelligent system approach to higher-dimensional classification of volume data (2005)】

【Featurelego: Volume exploration using exhaustive clustering of super-voxels (2019)】

【An Intelligent System Approach for Probabilistic Volume Rendering using Hierarchical 3D Convolutional Sparse Coding (2019)】

【Learning probabilistic transfer functions: A comparative study of classifiers (2018)】

\subsection{多变量体数据特征提取}
【Biclusters based visual exploration of multivariate scientific data visualization (2018)】

【Easyxplorer: A flexible visual exploration approach for multivariate spatial data (2015)】

【Guideme: Slice-guided semiautomatic multivariate exploration of volumes (2014)】

\subsection{时序体数据特征提取}
特征演化

【nested tracking graphs ((2017))】

【Dynamic nested tracking graphs (2019)】

特征追踪

【】


\section{基于深度学习的体数据分析方法/深度学习在体数据中的应用}
传统的体数据表达缺陷

深度学习方法

特征-based

【Flownet: A deep learning framework for clustering and selection of streamlines and stream surfaces (2018)】

【Insitunet: Deep image synthesis for parameter space exploration of ensemble simulations (2020)】

体数据-based

【A deep learning approach to selecting representative time steps for time-varying multivariate data (2019)】

【V2v: A deep learning approach to variable-to-variable selection and translation for multivariate time-varying data (2020)】

【Tsr-tvd: Temporal super-resolution for time-varying data analysis and visualization (2020)】

体素-based

【An intelligent system approach to higher-dimensional classification of volume data】

【A Cluster-Space Visual Interface for Arbitrary Dimensional Classification of Volume Data】


\subsection{基于图论的体数据表征学习}

树、图的进一步表征()

【】


\chapter{基于自然语言处理的体数据嵌入表征的学习与应用}

\section{引言}

\section{voxel2vec}
\subsection{Skip-Gram模型}
\subsection{负采样}
\subsubsection{适应性负采样}
\subsubsection{自取代学习}
\subsection{模型训练}

\section{应用}
\subsection{多变量体数据特征分类}
\subsection{关联性分析}
\subsubsection{时变体数据关联性分析}
\subsubsection{组集体数据关联性分析}
\subsubsection{时变组集体数据关联性分析}

\section{讨论}
\subsection{超参数分析}
\subsection{效率分析}
\subsection{voxel2vec模型分析}
\subsection{单变量体数据上的验证}
\subsubsection{比较}
\subsubsection{负采样策略分析}
\subsection{与已有工作的对比}
\section{本章小结}

\chapter{数值-变量嵌入表征的学习与应用}
\section{引言}
\section{ScalarGCN}
\subsection{ScalarGraph}
\subsubsection{结构构建}
\subsubsection{节点特征构建}
\subsection{ScalarGCN搭建}
\subsection{模型训练}
\section{应用}
\subsection{单变量特征分类}
\subsection{多变量相关性分析}
\subsubsection{时变体数据关联性分析}
\subsubsection{组集体数据关联性分析}
\subsubsection{时变组集体数据关联性分析}
\subsection{结果分析}
\section{讨论}
\section{本章小结}

\chapter{基于图卷积网络的半监督体数据特征分类方法}
\section{引言}
\section{超体素图的生成}
\section{模型搭建}
\section{应用分析}
\section{讨论}
\section{本章小结}


\chapter{总结与展望}
\subsection{本文工作总结}
\subsection{未来工作展望}

\begin{figure}[htbp]
    \centering
    \includegraphics[width=.3\linewidth]{example-image-a}
    \caption{\label{fig:fig-placeholder}图片占位符}
\end{figure}
