\cleardoublepage
\chapternonum{摘要}

随着科技的进步和计算设备能力的提升,设备采集及模拟仿真的体数据被广泛应用于气象模拟、计算流体力学、燃烧模拟、医学等科学领域。这些体数据通常利用多个变量、多个参数、多个时间步来描述同一空间区域的不同物理或化学特性,因此,体数据通常具有大规模、高分辨率、多变量等特点。体数据能够表征丰富的数据特征,因此,体数据分析方法已被广泛应用到辅助领域用户深入理解科学过程,验证假设,甚至发现新的现象和规律等。然而体数据分析仍然具有大量问题与挑战。首先,体数据具有大量的冗余信息,例如背景、噪音等信息容易干扰关键信息的提炼。其次,变量、数值、特征之间的关联性隐蔽,不同层次数据的交互关系难以全面建模。另外,特征在体数据中具有复杂的内在结构,难以通过传统方式提取与可视。为此,本文旨在研究高效的体数据分析方法,针对冗余大、关联性隐蔽、内在结构复杂等问题,以体数据为主要研究对象,围绕体数据嵌入表征、相关性分析、特征提取等问题展开研究,以辅助领域专家及用户深层次理解和挖掘体数据。

\cleardoublepage
\chapternonum{Abstract}